% 附录
%% 依赖设置组:science_general_config
%% 需要assets:无

\part{附录}

\setcounter{part}{0}
\resetfont
\renewcommand{\thesection}{\Alph{section}}
\titleformat{\section}{\Large \SimHei}{附录 \thesection}{1em}{}
\renewcommand{\thesubsection}{\thesection-\arabic{subsection}}
\def\FAQvspace{0pt}

\section{常见问题(FAQ)\label{content::FAQ}}

\FAQQ{\LaTeX 到底如何发音?}

\FAQA{这要追溯到它的词源。\TeX 一词来源于\texttau\textepsilon\textchi,这是一个希腊语中的词根,与科学和艺术有关,发音类似technology中的tech音节(事实上tech这个词根也与这一希腊语词根有关),可读作\textipa{[t\textepsilon k]}或\textipa{[t\textepsilon x]}(注意不能读作\textipa{[t\textepsilon ks]})。前缀la-则发\textipa{[le\textsci]}或\textipa{[l\textscripta:]}。合起来,则\LaTeX 的发音接近于lay-tek或lay-tech。}\\[\FAQvspace]

\FAQQ{为什么我无法像本文源码中那样使用\addbs{addbs}等宏?}

\FAQA{源码中前导区内以\addbs{newcommand}声明的宏都不是自带宏。如果您也希望使用类似宏,需要自己加入自己的文档中。本文中未在前导区以\addbs{newcommand}声明的宏都是文档类型内置宏或来自宏包,如果是其他宏无法正确编译,那很可能是宏包安装不正确、文档类型不匹配或宏包引用不正确。}\\[\FAQvspace]

\FAQQ{如何安装宏包?}

\FAQA{不同的\LaTeX 发行版的具体操作不一样,请查阅您所使用发行版的文档或咨询技术人员。本文使用的\MiKTeX 则附带包管理器,遇到未安装的包时可能会自动安装它们或者向您索要许可,取决于您如何安装\MiKTeX。}\\[\FAQvspace]

\FAQQ{为什么生成文档没有乱码,源码却无法正确显示?}

\FAQA{本文源码采用UTF-8编码,笔者建议读者们也尽可能采取UTF-8编码。Windows平台所广泛采用的GBK2312编码实际上并不是很常见。如果出现了用自己的软件打开pdf文件没有乱码,打开源码却是乱码,那么很有可能是编码问题,使用专门工具将源码从UTF-8转到你的平台所使用的编码就可以了。}\\[\FAQvspace]

\FAQQ{为什么我的文本编辑器打开源码后,产生了多于预期的空行?}

\FAQA{如果“多余”的空行每处只有一行,那么其实是正常的。\LaTeX 语法中,源码需要换两行(留一行空行)才会在生成的pdf文档里换行。但如果你的文本编辑器显示了每处两个空行,那就是不太正常的现象,但也不必过于在意;这可能是因为源码是CRLF格式换行,但是你的文本编辑器是LF换行。}\\[\FAQvspace]

\FAQQ{为什么我的\LaTeX 报告错误“无法使用\pdfLaTeX 编译”?}

\FAQA{有些宏包不提供对\pdfLaTeX 的支持,所以会人为引入这个错误。强行使用\pdfLaTeX 编译只会让情况更糟,我们建议您尝试改用\XeLaTeX 并尽可能避免使用和\XeLaTeX 不兼容的宏包。}\\[\FAQvspace]

\FAQQ{源码原本应当是引用的部分,为何出现问号?}

\FAQA{\addbs{ref}引用的部分没有对应的\addbs{label}就会引发这个现象。检查你的include结构关系是否正确,有无遗漏,以及引用时有没有出现typo。}\\[\FAQvspace]

\FAQQ{我在尝试编译这份文档的源码,为什么我的编译器报告找不到图片?}

\FAQA{该文档原文确实包含了一些图片,放在与\LaTeX 源码同一目录下的img文件夹。如果您编译该文档的位置没有img文件夹或img文件夹内缺乏应有的文件,就会出现问题。}\\[\FAQvspace]

\FAQQ{我在尝试使用CJK搭配\pdfLaTeX 输出中文页眉并在前导区设置了标题格式,为什么没有效果?}

\FAQA{如果您确定要使用\pdfLaTeX,请在全文第一次使用中文之前进行CJK环境声明。如果前导区也有中文输出,那么只需要在前导区第一次出现中文字符之前声明一个空的CJK环境即可。}\\[\FAQvspace]

\FAQQ{我的编译器报告“无法打开文件”,这是为什么?}

\FAQA{有两种可能,其一是没有文件写入权限,其二是文件已被占用打开。前者可以通过文件权限标志来判断,此外您也需要判断文件位置是不是加了写保护。确定不是没有写入权限的话就基本可以确定是另一个进程已经以占用方式打开了该文件,只要把占用者关掉就可以了。通常而言,占用者主要是各类阅读器,比如Adobe Reader。}\\[\FAQvspace]

\FAQQ{我的\LaTeX 编译器报告的错误所在行数比我的文档总行数还多,是为什么?}

\FAQA{很可能是宏包内部问题,尝试调换宏包包含顺序,或者再编译一次,有时可以解决问题。如果还不行,有可能是宏包彼此之间的兼容性问题,也可能是宏包和\LaTeX 发行版不兼容。遇到兼容性问题是很难自己解决的,请咨询宏包发布者。}\\[\FAQvspace]

\FAQQ{\LaTeX 经常报出晦涩难懂的错误报告,是我的问题吗?}

\FAQA{一般而言,出错是用户的问题,但晦涩难懂的错误报告不是。\LaTeX 是基于宏的语言,相信计算机工程师们都知道一件事情,就是但凡有什么东西跟宏沾上了关系,它的底层原理就会变得极度复杂。\LaTeX 形式上的简便易用正是建立在底层纷繁复杂的宏上面的,这意味着一旦您写的\LaTeX 文档出现语法问题,\LaTeX 会在宏展开到底层时才能真正发现错误,正是这点导致了\LaTeX 的编译报错信息往往极其晦涩难懂。如果您无法接受这种违反人类习惯的报错信息,您可以去尝试阅读和书写C++的宏和模板,尤其是STL代码和Boost库代码。那样您就会明白,\LaTeX 报错的晦涩程度在计算机工程领域真的不值一提(笑)。当然这不是在批评C++,为了让最终用户得到方便,上层开发者不得不写很多连同代码和错误报告(如果最终用户的使用方式有误的话)都极其晦涩难懂的代码,这就是一门语言为了得到通用性和易用性不得不付出的代价,也就是极其难以读懂的底层错误报告。其实耐着性子读完\LaTeX 发行版给出的错误报告的话,揪出实际出错位置还是很容易的。\\说句题外话,C++不同编译器提供的错误报告可读性也不同,受不了GCC的yacc技术提供的报告的话不妨尝试Clang/LLVM。}\\[\FAQvspace]

\FAQQ{\LaTeX 报告了大量警告,有没有必要一个一个消除掉?}

\FAQA{我们建议您尽可能避免任何警告。然而作为一个排版软件,仅仅是字体问题就足够\LaTeX 报告上百个警告。只要您对最终输出的效果还算满意,那么关于字体的警告也并非十分重要。尽管如此,我们建议您使用专业工具以确定您的pdf是否真的使用了您所要求的字体,而不是某些替代品。Underfull \addbs{hbox}是另一个常见警告,通常是编译器无法判断合适的换行点造成的,如果您认为当前排版结果合适就可以忽视它。}\\[\FAQvspace]

\FAQQ{如何给我的pdf嵌入字体?}

\FAQA{通常而言嵌入字体不是\LaTeX 的工作,它只负责输出一份文档。如果您希望嵌入所使用的字体,一般的解决方案是使用第三方专业工具。如果您使用的字体不是太偏门,那么实际上\MiKTeX 就会帮您嵌入多数字体。}\\[\FAQvspace]

\FAQQ{我对计算机工程并不感兴趣,\LaTeX 适用于我吗?}

\FAQA{只要您不怕麻烦并且追求比Word更好的用户体验,那么无论您的工作方向是什么,\LaTeX 都很适合。文学作品的排版,算法著作,化学实验报告,物理学论文,甚至是漫画书,都可以用\LaTeX 来排版。本文主要面向计算机工程领域进行介绍,但这不代表计算机工程领域是\LaTeX 的唯一方向。}\\[\FAQvspace]

\FAQQ{我可以用\LaTeX 来写作业或论文吗?}

\FAQA{一般而言当然可以,不过具体看贵单位的要求如何。有些机构会强制要求你使用\LaTeX 的模板撰写论文,毕竟\LaTeX 比Word更加容易确定格式。有些学校也允许用\LaTeX 写论文,相信用户很容易体会到,用\LaTeX 写论文远比用Word方便,除非你从来没用过\LaTeX。笔者个人强烈建议每所学校都引入\LaTeX 作为标准文档工具。}\\[\FAQvspace]

\FAQQ{作为个人开发者,我如何确定素材的法律风险?}

\FAQA{一般而言,我们推荐使用有协议可查的素材,无论是图片、字体、代码、商标还是文字作品。这几类都是常见的高民事诉讼风险素材。对于各类素材,尽量寻找明确标注版权的网站,以及个人或组织的官方网站并确定作品应当如何被使用,同时尽可能标注来源以降低受到诉讼的风险。有些正版软件附带一些素材版权,如果您在使用这些软件,就可以在其中免费使用相应素材。(请注意:您购买了附带素材A版权的正版软件B,不代表您在同一台机器上的软件C也能合法使用素材A。)多数情况下 ,素材会有附带的使用条款或协议,我们建议您在发布前仔细审阅这些法律文件(特别地,不对第三方发布的文件一般也没有法律风险);应当特别提醒的是,不同语言的翻译版本含义出现冲突时应以原文为准,因此我们认为有必要完整阅读文件原文,而不是翻译。有些协议尽管法律效力不明确,如尚无相关司法解释和判例的,我们也仍然不建议违反它们,除非您不介意自己和自己的产品被作为反面教材被挂起来批评并被各大主流厂商排斥。最好的选择是只使用开放的材料和软件以规避法律风险,如果一定要使用法律风险高的素材,我们建议您在发布前仔细审阅相关法律文件,并在文件上随附声明。用于商业用途则务必咨询律师,除非您想和大厂商的法务部门来场法庭辩论。}
