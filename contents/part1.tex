% 正文第一部分
%% 依赖设置组:science_general_config
%% 所需assets:无

\part{前言:关于\LaTeX}
本文使用\LaTeX 写就,以本文自身使用的代码为例,展示和讲解\LaTeX 基本语法和在计算机工程领域撰写文章的常见用途用法。本文应当连同源码和一份附带的img图片资源文件夹发布,若没有,请向发布者索要,但必须在尊重协议的基础上。

\section{认识\LaTeX 和\TeX}

\LaTeX 和\TeX 都是用于排版的语言。\TeX 及其发展出的语言都是指令驱动的,用于生成pdf文件。这些语言的基础都在于宏控制序列,开发者可以通过编写代码来进行精确、高度自动化同时高度可定制的排版任务。无论是字体字号还是段落格式,随笔文集还是论文,都可以用\LaTeX 中高度集成化的指令设计解决。此外,\LaTeX 可以自动为您的各种题注进行编号,并方便地在后文中引用,这样您就不需要在编号变动后一个个手动改动引用了,这也意味着出错的几率大大降低。

\TeX 的出现很早,可追溯至HTML标记语言出现之前。\LaTeX 继承和发展了\TeX,可以简单地认为\TeX 相对“朴实无华”而\LaTeX 内置了众多常用宏。对本文读者而言只需要知道一件事:\LaTeX 相比一般的\TeX 更加易用,适合快速上手。本文接下来只讨论\LaTeX 而不再讨论\TeX。

\section{\LaTeX 能做到的事情,以及选择\LaTeX 的理由}

\LaTeX 在排版工作上几乎无所不能。使用社区提供的宏包,它能完成几乎任何常见排版需求。

对选择\LaTeX 顾虑最大的人群恐怕是Microsoft Office Word的用户。诚然Markdown和HTML\&CSS都可用于排版,但是Markdown和HTML\&CSS的业务还是主要集中于网络界面。Markdown也可以做文档排版,但是就笔者的经验而言,Markdown使用上虽然简易,但是很多自定义功能实现反而比\LaTeX 复杂一些。而在离线程序文档中,Markdown写就的README和\LaTeX 就更没有什么比较意义了。所以,笔者并不认为Markdown和\LaTeX 有太多的业务重叠。也有读者可能会使用方正和Adobe的产品,但那些产品通常是收费的(当然,Microsoft Office也是收费的。您也可以使用某些软件的免费版,但相比之下\LaTeX 中不存在免费版比收费版更差的情况),而且主要面向专业的排版厂商,个人开发者使用那种体量的企业级产品实在是没有必要。我们这里主要阐述\LaTeX 相比Microsoft Office Word的优劣。

相比于Microsoft Office Word的GUI为核心的设计,在\LaTeX 下您一般不需要特别考虑把图片放在哪,\LaTeX 会尝试自己解决这件事。如果对它的工作不满意,也可以通过参数自行指定。相比Word下需要点来点去拖来拖去,\LaTeX 只需要您已经写好的代码就能得到最终结果,要修改效果也只需要简单地改写代码,不需要打开额外的属性页面。

\LaTeX 另一大亮点是集成包管理系统,只需要引入几个包敲几行代码就能生成漂亮的数学公式或是化学方程式,物理论文或是生物论文自然也可以。如果需要插图,\LaTeX 也有一些常见的绘图软件,只要需要的示意图不是太复杂,您就能用\LaTeX 把它画出来。相比于传统讲义设计中繁琐的示意图绘制、公式插入和特殊符号插入,您在\LaTeX 下只需要几行代码。当然,这些代码需要用户自行编写。

传统计算机工程领域讲义编写中,代码高亮常常是通过截图或第三方工具实现的。这一切在\LaTeX 社区提供的宏包面前都黯然失色。在\LaTeX 下,您所需要的只是引入包,设置高亮格式,把代码复制进来。如果愿意,您甚至可以自己实现一门语言及其在\LaTeX 下的高亮。哪怕拥有yacc工具的用户也需要自己编写代码才能实现这样的自定义语言高亮,但在\LaTeX 下,用户只需要使用提供好的工具,甚至没有必要真的把这门语言实现出来。

然而,\LaTeX 是诞生于CLI时代的工具(当然也有IDE,但底层仍然是CLI),对于使用惯了Windows提供的图形化操作的用户而言,\LaTeX 可能不是太美好。而且,\LaTeX 并不是所见即所得的工具,所以您需要不断改写和编译您的源码才能得到期望的文档。但是,如果您真的是本文所面向的计算机领域工程师,我有理由相信您是乐于折腾手头的工具和设备的geek——否则恐怕不会选择这个行业。如果果真如此的话,有理由相信\LaTeX 能成为您今后的最佳伴侣。

\section{\LaTeX 的开发工具链及资料指南}

中文社区最常见的\LaTeX 开发工具是\CTeX,但笔者在使用\CTeX 的过程中遇到了不小的麻烦,主要是附带的\MiKTeX 版本太旧,无法用于更新宏包。笔者的建议是使用官网的\MiKTeX 进行\LaTeX 的编辑。如果您愿意,也可以安装不带\MiKTeX 的\CTeX 包,再单独安装\MiKTeX,这样可以有效避免宏包无法更新的问题。

\MiKTeX 和\CTeX 本质上是提供了开发环境,实际编译工作是附带的工具链来做的。比较常见的工具链有\pdfLaTeX、\XeLaTeX、\XeTeX 等,笔者建议中文用户使用\XeLaTeX。本文就是使用\XeLaTeX 编写的。

您可以从清华大学开源软件站\footnote{\url{https://mirrors.tuna.tsinghua.edu.cn/}}获取\CTeX 套装,而德语\TeX 用户组维护的网站\TeX doc Online\footnote{\url{http://texdoc.net/}}处可以获取\LaTeX 常见宏包的文档。此外,stackexchange也有一个专门的\TeX 论坛\footnote{\url{https://tex.stackexchange.com/}},可以在那里查询到常见问题的解决方案,还有一个称为\href{https://stackovernet.com/}{StackOvernet}的论坛也有大量技术资料。Comprehensive \TeX\;Archive Network\footnote{\url{http://ctan.math.illinois.edu}}同样提供大量可供查阅的文档,该网站也能下载到各种宏包。\LaTeX 论坛\footnote{\url{https://latex.org/forum/}}也有一些常见问题的解答。\MiKTeX 则可以从\MiKTeX 官网\footnote{\url{https://miktex.org/}}获得。

\LaTeX 包有相当一部分是社区维护的,除非您能联系到包的作者本人,否则不太可能获取客服技术帮助什么的。\LaTeX 宏包因此会出现各种兼容性问题,请注意自行查阅材料确定哪些包可以混用,哪些不可以。此外,\MiKTeX 用户应当了解,安装到计算机上的\MiKTeX 已经附带了包管理器,可以用于从网络上下载包并自动安装。