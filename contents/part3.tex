% 正文第三部分
%% 依赖设置组:science_general_config
%% 需要assets:无

\part{\LaTeX 基本功能}

\section{基本概念}

\subsection{控制序列和宏}

\subsubsection{控制序列和宏的概念}

宏是历史较长的语言中极为常见的一种特性。它诞生于早期编程语言理论不完善、工具不充足的时期,是一种基于文本进行处理的技术,简单说就是对文本进行模式替换或者根据模式重复一定动作。尽管宏在今天的开发者看来相当落后,时至今日,人类社会中仍有巨量软件和文档的生成依赖于宏或者宏驱动的语言。

在\LaTeX 中,宏和控制序列是很接近的概念,都是以反斜杠(\textbackslash)开始、转义后具有一定特殊含义、完成一定特殊功能的字符串。有些控制序列带有@符号,其语法和功能类似于面向对象编程中对对象内数据域的引用,但普通\LaTeX 文档中只能通过特殊手段启用这类序列;这类控制序列通常只有在宏包编写内或利用宏包底层接口封装新宏才用得到。宏则一般特指\addbs{def}控制序列产生的新的可用宏。

本文中不对宏和控制序列做具体区分。

\subsubsection{定义新宏}

\LaTeX 中的宏通过\addbs{def}或\addbs{newcommand}定义。有些语言中同名宏会覆盖之前的定义,但是\LaTeX 中同名宏通常会引发编译错误,需要使用\addbs{renewcommand}覆盖旧有定义。

本文源码中的\addbs{textlb}宏(lb表示line break)就是通过\addbs{newcommand}定义的:

\begin{lstlisting}[style = latex_texworks, numbers = none]
\newcommand\textlb{\textbackslash \textbackslash}
\end{lstlisting}

这是不需要参数的宏,宏名后只需要跟一对花括号指定宏内容。编译器处理对宏的引用时,会先按定义时指定的宏内容对宏进行原地展开(类似于C,只会展开完整token,而不会只展开半个token),然后进一步处理展开后的内容。

定义需要参数的宏时,宏名后首先跟一对方括号指定参数数目,再跟一对花括号指定宏内容。宏内容中以井号(\#)对参数按编号进行引用,例如\#1表示第一个参数,\#2表示第二个参数。举例而言,有一个这样的宏定义:

\begin{lstlisting}[style = latex_texworks, numbers = none]
\newcommand\addbs[1]{\textbackslash #1}
\end{lstlisting}

那么\addbs{addbs}\{foo\}将会被展开为\addbs{textbackslash} foo(首个参数就是foo),进一步编译后得到\addbs{foo}。如果需要多个参数,在引用宏时需要为每个参数都加一对花括号,但定义则与只有一个参数的宏类似,最外层只有一对花括号。例如,假若有这样一个宏定义:

\begin{lstlisting}[style = latex_texworks, numbers = none]
\newcommand\foo[2]{$#1_{#2}$}
\end{lstlisting}

那么使用\addbs{foo}\{a\}\{1\}对其进行的引用将会被展开为\$a\_1\$,这是\LaTeX 中的数学环境语法,显示效果为$a_1$。

其他宏包里的宏通常都是使用类似的语法传递参数,但也有些宏可以通过方括号传递参数,或者传递以逗号分割的参数列表。本文仅对\LaTeX 编写做简要介绍,这类进阶语法按下不表。

\subsection{文档类型}

\LaTeX 支持若干种预设的文档类型。不同文档类型的页面尺寸、默认字体、字号等等都有可能不同,当然,也可以显式指定参数来实现与预设不同的样式效果。很多.tex主文件第一行就是\addbs{documentclass}控制序列,我们使用该控制序列来指定当前文档的类型。正常而言,一组用于生成单个pdf文件的\LaTeX 源码应该恰好具有一个\addbs{documentclass}声明。

不同文档类型支持的控制序列也不同,这点需要格外注意。因此,当您试图在论坛上提问时,应当说明您正在使用或希望使用何种文档类型。

\subsection{导言区和正文}

用于生成pdf的源码中总是有一个\addbs{begin}\{document\}和\addbs{end}\{document\}括起来的区域,这个部分被称为正文。正文前的部分则称为导言区,导言区内通常是不直接参与文本生成的控制序列,或者输出到全局、不能手动直接修改的内容,例如数学环境的定义和提示文字、页眉页脚等,\addbs{documentclass}也不直接参与文本生成,而只是指定正文的某些样式。

\subsection{计数器}

\LaTeX 主要有两大类计数器:序号计数器和控制计数器,后者主要用于控制每页上浮动结构数量、目录深度等,前者则用于记录和显示序号。本文只阐述序号计数器。

\LaTeX 中有各种各样的控制序列用于产生具有层次的结构,主要是各级标题;下文将会详细介绍多级标题。枚举列表、数学公式、脚注边注等也有自己的多级计数器。序号计数器的初始值通常为0,也可以用\addbs{setcounter}\{计数器名称\}\{数值\}在任何时候进行赋值(会将跟随其编号的低级计数器置零)。\addbs{stepcounter}\{计数器名称\}将指定计数器的值加上1。多数宏包提供的序号计数器会在通过关联的控制序列被引用后自动步进。因此,如果你将计数器初始化为1,那么有关序号会从2开始编号,而不是1。

计数器一般附带一个控制序列,即“\addbs{the}计数器名称”,可以用于获得计数器的值。通过该方式获得的计数器的值可以是罗马数字、阿拉伯数字等,取决于它被如何设置。\addbs{value}\{计数器名称\}可以获取计数器值的阿拉伯数字形式,这个特点可以用来将序号输出格式设置为本来不接收计数器参数的数字格式化控制序列。

可以通过\addbs{newcounter}\{计数器名称\}[计数器跟随项]设置新的计数器,但这不在本文探讨的范围内。

\section{换行和换页}

很容易发现,在源码下换行后,编译出的pdf文件对应位置往往并没有换行(这就是为什么本文源码有那么多空行)。\LaTeX 下最常见的两种换行方式分别是使用换行符号\textlb,以及额外空一行。这两种方式都可以用于换行。本文中多数地方使用源码中空行来换行,但细心的读者也许已经发现,本文源码中document环境的title就用了\textlb 来换行。

\begin{lstlisting}[style = latex_texworks, numbers = none]
\title{\bf \docTitle \\ A Brief Chinese Tutorial on \LaTeX}
\end{lstlisting}

使用\textlb 换行还有一个技巧,如果附带一对方括号,方括号内部写入一个长度,那么当前换行将会产生额外的空白。本行结尾有一个\textlb[3mm]来展示这个特点。\\[3mm]

如你所见,产生了一段竖直方向上的空白。这段空白包含一个空行、两个默认行距外加3mm的空白。此外,也有时使用\addbs{newline}进行换行。

换页通常使用\addbs{newpage}或\addbs{clearpage}来实现。不同之处在于,前者只是结束当前页,后者还会把图表等浮动结构输出。\addbs{include}(用于包含另一文件)也会先另起一页再输出被引用文件的内容。

\section{通用格式控制}

\LaTeX 允许控制文字的字体、字号等信息,其中多数功能需要配合宏包提供的附加功能,如表\ref*{table::font_control_sequences}所示。数学环境有独立的文字样式设定。

\begin{center}
	\begin{longtable}{C{2.5cm}C{2.5cm}C{4cm}C{2cm}C{1.5cm}}
		\hline
		宏 & 功能 & 参数 & 效果 & 来源 \\
		\hline
		\addbs{textup} & 无效果字体 & \{要显示的文字\} & \textup{text} & \LaTeX \\
		\addbs{textit} & 意大利斜体字体 & \{要显示的文字\} & \textit{text} & \LaTeX \\
		\addbs{textsl} & Slanted斜体字体 & \{要显示的文字\} & \textsl{text} & \LaTeX \\
		\addbs{textsc} & 小写转大写后输出 & \{要显示的文字\} & \textsc{text} & \LaTeX \\
		\addbs{textrm} & 罗马字体族 & \{要显示的文字\} & \textrm{text} & \LaTeX \\
		\addbs{textsf} & 无衬字体 & \{要显示的文字\} & \textsf{text} & \LaTeX \\
		\addbs{texttt} & 打印字体 & \{要显示的文字\} & \texttt{text} & \LaTeX \\
		\addbs{newfontfamily} & 指定字体格式 & \{控制序列\}\{字体名\} & \newfontfamily{\fontcn}{Courier New} \fontcn N/A & fontspec \\
		\addbs{setCJKfamilyfont} & 导入中日韩(CJK)字体,赋予一个别名 & \{别名\}\{字体名\} & \setCJKfamilyfont{kf}{楷体} N/A & ctex \\
		\addbs{CJKfamily} & 设置中日韩(CJK)语言环境字体 & \{字体名称\} & \CJKfamily{kf} 文字 & ctex \\
		\addbs{textbf} & 文字加粗 & \{要显示的文字\} & \textbf{text 文字} & \LaTeX \\
		\addbs{underline} & 添加下划线 & \{要显示的文字\} & \underline{text 文字} & \LaTeX \\
		\addbs{textcolor} & 彩色文字 & [色彩空间]\{浮点数色彩值\}\{要显示的文字\} & \textcolor[rgb]{1, 0, 0}{text 文字} & xcolor \\
		\addbs{tiny} & 最小的预设字号 & N/A & \tiny text 文字 & \LaTeX \\
		\addbs{footnotesize} & 脚注标准字号 & N/A & \footnotesize text 文字 & \LaTeX \\
		\addbs{small} & 小号字体 & N/A & \small text 文字 & \LaTeX \\
		\addbs{normalsize} & 默认字号 & N/A & \normalsize text 文字 & \LaTeX \\
		\addbs{large} & 较大的字号 & N/A & \large test 文字 & \LaTeX \\
		\addbs{Large} & 介于large与LARGE间的字号 & N/A & \Large text 文字 & \LaTeX \\
		\addbs{LARGE} & 介于Large与huge间的字号 & N/A & \LARGE text 文字 & \LaTeX \\
		\addbs{huge} & 介于LARGE与Huge间的字号 & N/A & \huge text 文字 & \LaTeX \\
		\addbs{Huge} & 最大的预设字号 & N/A & \Huge text 文字 & \LaTeX \\
		\addbs{zihao} & 中国人习惯的汉字字号 & \{字号\} & \zihao{3} 三号字 & \LaTeX \\
		\hline
		\caption{文本文字样式控制序列}\label{table::font_control_sequences}
	\end{longtable}
\end{center}

补充说明几点:

\begin{itemize}

\item \addbs{newfontfamily}本身并不直接参与文本生成。不难注意到该控制序列的第一个参数也是一个控制序列,我们使用它即可将新控制序列绑定的字体应用到跟随其的文本。您可能已经注意到,表格中有一个“N/A”项的字体与其他的不太一样。

\item 读者或许会发现,color包也能使用\addbs{textcolor}宏;本文中引用的许多包彼此之间具有包含关系,笔者只在正文中列出笔者最常用的包,其中有些并不是其原始来源。此外,对照源码不难发现:应用字体的控制序列只会影响其后附近的文字,而不能改变全局字体设置。实际上,如果把字体设置控制序列放在正文最外层,其造成的字体改变确实是全局的,直到重新指定另一个字体;这里不能影响全局是因为我们是在单元格环境内进行的设置,离开了这个单元格,单元格内的设置就会失效。

\end{itemize}

\caution{别忘了,多数字体需要先通过控制序列引入才能正常使用。此外,有些字体还需要fontenc包。}\resetfont

\section{层次结构和标题}

\LaTeX 使用的层次结构可以十分多样。不同文档类型常见的搭配也不同,比如article类型经常搭配part而不是chapter,但无论是哪一个,其本质是大同小异的。\LaTeX 下任何段落层次结构都以类似方式声明,即“\addbs{结构类型名称}\{标题\}”的形式。本文仅介绍article类型,其他很多类型大致相似,但一般都有一些特有控制序列。

在article文档类型下,常见的结构类型有part、section、subsection、subsubsection。使用\addbs{section}\{标题\}就可以生成一个以“标题”为标题的小节。如果加一个星号即\addbs{section}*\{标题\},则生成的小节不会带有编号,多数目录工具也会在目录中跳过不带编号的小节。其他层次也都有类似的特性。不同的是,part是在全局上独立编号的,section在每个part内独立编号,subsection跟随section编号,subsubsection跟随subsection编号。例如,part 2,section 3,subsection 4的编号默认会显示为3.4。本文由于采用了一些特殊设定,section跟随part编号并显示,所以会显示为2.3.4。

\subsection{标题样式}

\subsubsection{编号数字样式}

\LaTeX 支持显示不同形式的数字,包括大小写字母、大小写罗马数字和阿拉伯数字。如果引用了ctex宏包,还可以显示汉语数字、大写汉语数字、天干地支等。

设置编号数字的样式非常简单,只要使用\addbs{renewcommand}重新定义计数器宏即可。重定义宏的内容通常是\addbs{格式控制序列}\{计数器名称\}的形式,默认的、有关序号格式的控制序列有:

\begin{figure}[h]
	\centering
	\begin{tabular}{cc}
		\hline
		宏 & 内容 \\
		\hline
		\addbs{alph} & 小写字母 \\
		\addbs{Alph} & 大写字母 \\
		\addbs{arabic} & 阿拉伯字母 \\
		\addbs{roman} & 小写罗马数字 \\
		\addbs{Roman} & 大写罗马数字 \\
		\hline
	\end{tabular}
\end{figure}

\subsection{改变标题格式\label{content::title_format_tutorial}}

本文介绍的改变标题格式的方法以titlesec的\addbs{titleformat}为例,这是最常见的方案。

考虑到引用位置的需求,各级标题中最重要的部分就是编号。这里我们先明确计数器的概念。\LaTeX 中计数器一般指的是part、section等默认计数器(你也可以生成自己的计数器,但本文不讨论这点),它们表示的就是对应结构的当前计数值。例如,本小节当前的计数器(subsection)值为\ref{content::title_format_tutorial}。你也可以自行修改计数器的值,但本文也不讨论这点。

\LaTeX 下,很多计数器具有一个以the为前缀的控制序列,用于控制编号输出样式。例如,\addbs{thepart}表示当前part序号。本小节的位置可以用\addbs{thesubsection}表示\footnote{正在对照源码的读者可能注意到了上文使用了\addbs{ref}。\addbs{ref}是用于全文引用某位置的语法,但是\addbs{thesubsection}永远只能指示当前位置。这就是二者的区别。}。

\begin{tabular}{|c|c|}
\hline
宏 & 在该文档本位置的值 \\ \hline
\addbs{thepart} & \thepart \\ \hline
\addbs{thesection} & \thesection \\ \hline
\addbs{thesubsection} & \thesubsection \\ \hline
\end{tabular}

这里特别强调一点:计数器名称不带反斜杠(\textbackslash),但是对应的宏是有的。\LaTeX 的一切宏都以反斜杠开头以标识出它是个宏。除了特定宏指令参数外,part都不会被当作计数器名称,而是普通的纯文本(plain text)。

计时器对应宏的显示样式可以用\addbs{renewcommand}\{样式控制序列\}来改变。如果你希望使用罗马数字作为section序号,那么可以这样做:

\begin{lstlisting}[style = latex_texworks, numbers = none]
\renewcommand}{\Roman{section}}
\end{lstlisting}