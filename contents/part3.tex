% 正文第三部分
%% 依赖设置组:science_general_config
%% 需要assets:无

\part{\LaTeX 基本功能}

\section{基本概念}

\subsection{控制序列和宏}

\subsubsection{控制序列和宏的概念}

宏是历史较长的语言中极为常见的一种特性。它诞生于早期编程语言理论不完善、工具不充足的时期,是一种基于文本进行处理的技术,简单说就是对文本进行模式替换或者根据模式重复一定动作。尽管宏在今天的开发者看来相当落后,时至今日,人类社会中仍有巨量软件和文档的生成依赖于宏或者宏驱动的语言。

在\LaTeX 中,宏和控制序列是很接近的概念,都是以反斜杠(\textbackslash)开始、转义后具有一定特殊含义、完成一定特殊功能的字符串。有些控制序列带有@符号,其语法和功能类似于面向对象编程中对对象内数据域的引用,但普通\LaTeX 文档中只能通过特殊手段启用这类序列;这类控制序列通常只有在宏包编写内或利用宏包底层接口封装新宏才用得到。宏则一般特指\addbs{def}控制序列产生的新的可用宏。

本文中不对宏和控制序列做具体区分。

\subsubsection{定义新宏}

\LaTeX 中的宏通过\addbs{def}或\addbs{newcommand}定义。有些语言中同名宏会覆盖之前的定义,但是\LaTeX 中同名宏通常会引发编译错误,需要使用\addbs{renewcommand}覆盖旧有定义。

本文源码中的\addbs{textlb}宏(lb表示line break)就是通过\addbs{newcommand}定义的:

\begin{lstlisting}[style = latex_texworks, numbers = none]
\newcommand\textlb{\textbackslash \textbackslash}
\end{lstlisting}

这是不需要参数的宏,宏名后只需要跟一对花括号指定宏内容。编译器处理对宏的引用时,会先按定义时指定的宏内容对宏进行原地展开(类似于C,只会展开完整token,而不会只展开半个token),然后进一步处理展开后的内容。

定义需要参数的宏时,宏名后首先跟一对方括号指定参数数目,再跟一对花括号指定宏内容。宏内容中以井号(\#)对参数按编号进行引用,例如\#1表示第一个参数,\#2表示第二个参数。举例而言,有一个这样的宏定义:

\begin{lstlisting}[style = latex_texworks, numbers = none]
\newcommand\addbs[1]{\textbackslash #1}
\end{lstlisting}

那么\addbs{addbs}\{foo\}将会被展开为\addbs{textbackslash} foo(首个参数就是foo),进一步编译后得到\addbs{foo}。如果需要多个参数,在引用宏时需要为每个参数都加一对花括号,但定义则与只有一个参数的宏类似,最外层只有一对花括号。例如,假若有这样一个宏定义:

\begin{lstlisting}[style = latex_texworks, numbers = none]
\newcommand\foo[2]{$#1_{#2}$}
\end{lstlisting}

那么使用\addbs{foo}\{a\}\{1\}对其进行的引用将会被展开为\$a\_1\$,这是\LaTeX 中的数学环境语法,显示效果为$a_1$。

其他宏包里的宏通常都是使用类似的语法传递参数,但也有些宏可以通过方括号传递参数,或者传递以逗号分割的参数列表。本文仅对\LaTeX 编写做简要介绍,这类进阶语法按下不表。

\subsection{文档类型}

\LaTeX 支持若干种预设的文档类型。不同文档类型的页面尺寸、默认字体、字号等等都有可能不同,当然,也可以显式指定参数来实现与预设不同的样式效果。很多.tex主文件第一行就是\addbs{documentclass}控制序列,我们使用该控制序列来指定当前文档的类型。正常而言,一组用于生成单个pdf文件的\LaTeX 源码应该恰好具有一个\addbs{documentclass}声明。

不同文档类型支持的控制序列也不同,这点需要格外注意。因此,当您试图在论坛上提问时,应当说明您正在使用或希望使用何种文档类型。

\subsection{导言区和正文}

用于生成pdf的源码中总是有一个\addbs{begin}\{document\}和\addbs{end}\{document\}括起来的区域,这个部分被称为正文。正文前的部分则称为导言区,导言区内通常是不直接参与文本生成的控制序列,或者输出到全局、不能手动直接修改的内容,例如数学环境的定义和提示文字、页眉页脚等,\addbs{documentclass}也不直接参与文本生成,而只是指定正文的某些样式。

\section{行、页和章节层次}

\subsection{换行和换页}

很容易发现,在源码下换行后,编译出的pdf文件对应位置往往并没有换行(这就是为什么本文源码有那么多空行)。\LaTeX 下最常见的两种换行方式分别是使用换行符号\textlb,以及额外空一行。这两种方式都可以用于换行。本文中多数地方使用源码中空行来换行,但细心的读者也许已经发现,本文源码中document环境的title就用了\textlb 来换行。

\begin{lstlisting}[style = latex_texworks, numbers = none]
\title{\bf \docTitle \\ A Brief Chinese Tutorial on \LaTeX}
\end{lstlisting}

使用\textlb 换行还有一个技巧,如果附带一对方括号,方括号内部写入一个长度,那么当前换行将会产生额外的空白。本行结尾有一个\textlb[3mm]来展示这个特点。\\[3mm]

如你所见,产生了一段竖直方向上的空白。这段空白包含一个空行、两个默认行距外加3mm的空白。此外,也有时使用\addbs{newline}进行换行。

换页通常使用\addbs{newpage}或\addbs{clearpage}来实现。不同之处在于,前者只是结束当前页,后者还会把图表等浮动结构输出。\addbs{include}(用于包含另一文件)也会先另起一页再输出被引用文件的内容。

\subsection{层次结构}

\LaTeX 使用的层次结构可以十分多样。不同文档类型常见的搭配也不同,比如article类型经常搭配part而不是chapter,但无论是哪一个,其本质是大同小异的。\LaTeX 下任何段落层次结构都以类似方式声明,即“\addbs{结构类型名称}\{标题\}”的形式。本文仅介绍article类型,其他很多类型大致相似,但一般都有一些特有控制序列。

在article文档类型下,常见的结构类型有part、section、subsection、subsubsection。使用\addbs{section}\{标题\}就可以生成一个以“标题”为标题的小节。如果加一个星号即\addbs{section}*\{标题\},则生成的小节不会带有编号,多数目录工具也会在目录中跳过不带编号的小节。其他层次也都有类似的特性。

\subsection{改变标题格式\label{content::title_format_tutorial}}

本文介绍的改变标题格式的方法以titlesec的\addbs{titleformat}为例,这是最常见的方案。

考虑到引用位置的需求,各级标题中最重要的部分就是编号。这里我们先明确计数器的概念。\LaTeX 中计数器一般指的是part、section等默认计数器(你也可以生成自己的计数器,但本文不讨论这点),它们表示的就是对应结构的当前计数值。例如,本小节当前的计数器(subsection)值为\ref{content::title_format_tutorial}。你也可以自行修改计数器的值,但本文也不讨论这点。

\LaTeX 下,很多计数器具有一个以the为前缀的控制序列,用于控制编号输出样式。例如,\addbs{thepart}表示当前part序号。本小节的位置可以用\addbs{thesubsection}表示\footnote{正在对照源码的读者可能注意到了上文使用了\addbs{ref}。\addbs{ref}是用于全文引用某位置的语法,但是\addbs{thesubsection}永远只能指示当前位置。这就是二者的区别。}。

\begin{tabular}{|c|c|}
\hline
宏 & 在该文档本位置的值 \\ \hline
\addbs{thepart} & \thepart \\ \hline
\addbs{thesection} & \thesection \\ \hline
\addbs{thesubsection} & \thesubsection \\ \hline
\end{tabular}

这里特别强调一点:计数器名称不带反斜杠(\textbackslash),但是对应的宏是有的。\LaTeX 的一切宏都以反斜杠开头以标识出它是个宏。除了特定宏指令参数外,part都不会被当作计数器名称,而是普通的纯文本(plain text)。

计时器对应宏的显示样式可以用\addbs{renewcommand}\{样式控制序列\}来改变。如果你希望使用罗马数字作为section序号,那么可以这样做:

\begin{lstlisting}[style = latex_texworks, numbers = none]
\renewcommand}{\Roman{section}}
\end{lstlisting}