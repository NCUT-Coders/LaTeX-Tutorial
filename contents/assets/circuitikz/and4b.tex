% 定义and4b(4脚与门)芯片
\makeatletter
\pgfdeclareshape{and4b}{
	\anchor{center}{\pgfpointorigin}
	\anchor{text}{\pgfpoint{-.5\wd\pgfnodeparttextbox}{-.5\ht\pgfnodeparttextbox}}
\savedmacro{\resize}{\setlength{\IClen}{\pgfkeysvalueof{/tikz/pin spacing}}\ifdim\IClen=0mm \setlength{\IClen}{\pgfkeysvalueof{/tikz/circuitikz/multipoles/pin spacing}} \fi}
	\savedanchor\icpina{\pgfpoint{-1\IClen}{1.5\IClen}}\anchor{in1}{\icpina}
	\savedanchor\icpinb{\pgfpoint{-1\IClen}{.5\IClen}}\anchor{in2}{\icpinb}
	\savedanchor\icpinc{\pgfpoint{-1\IClen}{-.5\IClen}}\anchor{in3}{\icpinc}
	\savedanchor\icpind{\pgfpoint{-1\IClen}{-1.5\IClen}}\anchor{in4}{\icpind}
	\savedanchor\icpine{\pgfpoint{1\IClen}{0\IClen}}\anchor{out}{\icpine} %% 设置针脚
	\savedanchor{\northeast}{\pgfpoint{1\IClen}{2\IClen}}
	\savedanchor{\southwest}{\pgfpoint{-1\IClen}{-2\IClen}} %% 这两行设置矩形大小
	\compassAnchor

	\foregroundpath{
		\pgfsetlinewidth{.1\IClen} %% 矩形线宽
		\pgfpathrectanglecorners{\southwest}{\northeast}
		\pgfusepath{draw}
		%% 绘制半圆形定位孔
		\pgfsetlinewidth{.06\IClen} %%% 半圆形定位孔线宽(定位孔用于指示针脚顺序,方便阅读说明)
		\pgfextracty{\tempy}{\northeast} %%% temp定位上方
		\pgfpathmoveto{\pgfpoint{-.2\IClen}{\tempy}} %%% 定位孔位置
		\pgfpatharc{-180}{0}{.2\IClen} %%% 定位孔绘制参数
		\pgfusepath{draw}
		%% 绘制针脚标记
		\pgftext[left, at = \icpina]{\pinLabelSize \,i1}
		\pgftext[left, at = \icpinb]{\pinLabelSize \,i2}
		\pgftext[left, at = \icpinc]{\pinLabelSize \,i3}
		\pgftext[left, at = \icpind]{\pinLabelSize \,i4}
		\pgftext[right, at = \icpine]{\pinLabelSize o\,}
	} %% 'foregroundpath'块结束
} %% 'pgfdeclareshape'(not4b)块结束
\makeatother
