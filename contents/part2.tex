% 正文第二部分
%% 依赖设置组:science_general_config
%% 所需assets:
%% contents/assets/LaTeX_TeXworks(LaTeX语法高亮)
%% contents/assets/GNU_Cpp20_DevCpp(C++20语法高亮)

\part{开始使用\LaTeX}

\section{你好,\LaTeX}

\LaTeX 的主要构成十分简单,第一行声明基本文档类型(\addbs{documentclass}),包含宏包(非必要,但没有宏包基本上做不了任何事情),前导区(preamble)声明必要的东西,最后开始写文档体。

\begin{lstlisting}[style = LaTeX_TeXworks, caption = Hello \LaTeX]
\documentclass{article}

% \begin 表示开始环境,与\end 配对
% document 表示文档体环境
% 百分号开始的内容一般是注释,但也有例外
\begin{document}
Hello \LaTeX!
\end{document}
\end{lstlisting}

这就是一个\LaTeX 的Hello world文档(不过输出的是Hello \LaTeX 而不是Hello world),在TeXworks界面上按下排版键\includegraphics{contents/assets/img/typeset_key.png}(默认快捷键\keys{Ctrl + T})即可看到编译结果。该例子中的源码只输出一页pdf,包含一行Hello world字样。它没有引用任何包。从这个例子中可见,\LaTeX 的显示内容是直接写在文档里面就好的,由控制序列(控制序列是它的正式名称,可以简单理解为指令)指定格式。

上述例子还无法输出中文。如果简单将Hello world文本更换为中文,你会发现,它什么也不会输出;编译器没有正确辨认中文字符。

\section{支持中文的\LaTeX 文档}

给文档加一个包即可正确向pdf文件写入中文。以下是\XeLaTeX 下的做法:

\begin{lstlisting}[style = LaTeX_TeXworks, caption = 你好,中文\LaTeX - \XeLaTeX]
\documentclass{article}
% 有了这个包,就能正确输出中文了
\usepackage{ctex}

\begin{document}
(@*\textcolor{bgcolor_gray}{ }*@)
你好,\LaTeX!
\end{document}
\end{lstlisting}

使用\pdfLaTeX,则需要使用另一个方案:

\begin{lstlisting}[style = LaTeX_TeXworks, caption = 你好,中文\LaTeX - \pdfLaTeX]
\documentclass{article}
\usepackage{CJK}

\begin{document}
\begin{CJK}{UTF8}{song}
(@*\textcolor{bgcolor_gray}{ }*@)
你好,\LaTeX!
\end{CJK}
\end{document}
\end{lstlisting}
以上代码即是令\LaTeX 支持中文的使用方法\footnote{如果您正在比对本文源码阅读该份pdf文件,您可能已经注意到了:上述代码片中的lstlisting环境中都加入了一行奇怪的代码,即(@*\addbs{textcolor}\{bgcolor\_gray\}\{ \}*@)而且呈现奇怪的缩进。这里加入一个没有在pdf文件中显示出来的控制序列是为了避免错误的换行。总之,读者目前只需要知道一件事:这个控制序列是为了修正listings包在引用代码片时和中文的不兼容引发的样式问题而加入的,而不是编写中文\LaTeX 文档所必要的控制序列。后文会简单讲解该控制序列的含义。至于缩进……\LaTeX 是一个不在乎缩进的语言,有些时候空行也无关紧要(但更多的时候,空行有特别意义),这里只好先牺牲一点样式效果,避免更大的样式问题。}。不过,笔者个人喜欢使用ctexart文档类型。

\section{初学者会用到的包}

\LaTeX 实际上是以宏\footnote{宏的概念可以理解为直接原地展开宏内容并执行操作。有程序设计基础的读者可以联想其他语言的宏概念来理解它。宏有利于大量减少代码量,不过相比于其他代码生成技术,宏有个显著缺点:可读性很差。我们建议工程师编写宏时辅以充足的文档说明。}为基础的语言,所以严格而言,“包”在这里应当叫做宏包。宏以\addbs{def}命令定义,此外,\addbs{newenvironment}、\addbs{newcommand}和\addbs{renewcommand}分别可以定义新环境、新宏和修改特定宏的作用。作为一篇简明教程,这里我们只进行“脚本小子”式的教学,讲述常见几个宏包的使用方法,而略去关于宏本身的内容。


\subsection{语言支持包}

对于中文用户,最常见的应该是CJK包和ctex包,这两个是中文支持包。此外,还有greektext希腊语包等其他语言支持。值得一提的是,有些院校要求子目录使用汉字编号,这可以用zhnumber包来完成,后面的章节会详细说明。还有一些包提供更多语言学特性,例如生成音标用的tipa包(IPA就是国际音标的缩写)。

用tipa包输出Hello \LaTeX 的音标为\textipa{[h\textschwa"l\textschwa\textupsilon][le\textsci t\textepsilon k]}\footnote{实际上\LaTeX 不仅这一种读法,详见\hyperref[FAQ]{附录FAQ}。}。

\subsection{理工科类宏包}

\renewcommand{\thefootnote}{[\arabic{footnote}]} % 这行用于临时改变脚注格式,不改变的话会造成混淆

主要是amsthm包,它提供数学环境和符号。常见的还有extarrows包,它提供特殊的箭头,允许在长箭头上附加文字,可用于书写无机化学方程式或核物理反应方程式。chemfig和mhchem则常用于书写有机化学方程式,SIunits包提供标准物理学单位。对于计算机行业工作者,最常见的功能大概是引用代码,这经常由listings包来完成。还有一个常见的叫做algorithm2e的宏包,用于生成伪代码。

下面以一些常见用途来举例子。

\begin{enumerate}

\item $ \hat{f}(\omega)=\int^{\infty}_{-\infty}f(t)e^{-i\omega t}dt $

这是傅里叶变换的基本公式。读者今天能用上电子产品不仅要感谢麦克斯韦等基础电磁理论的奠基人,还要感谢这个您也许看不懂的东西。不过如果您是学电子信息的,那就一定要弄懂这个东西,还要学好高等数学和复变函数。

\item \ce{_2^4He + _13^27Al -> _15^30P + _0^1n}

这是一个高中常见的\textalpha 粒子轰击铝的核物理方程式。使用\textalpha 粒子轰击铝箔后产生不稳定的\ce{^30P},这种放射性核素很快就会继续衰变。\ce{^30P}是最早从实验室制造出来的人工放射性核素之一。

\item \ce{CH3CHO + 2Ag(NH3)2OH ->[{水浴加热}] CH3COONH4 + 2Ag v + 3NH3 + H2O}

这个是银镜反应的化学方程式,我们在这里用它展示有特殊反应条件的方程式如何用\LaTeX 书写。

\item chemfig绘制的结构式:

\chemfig[atom sep = 2.5em]{[:30]*6(=-(--[:60](=[:120]O)-[,,1,1]NH>:[:60]*4(-(<[:-60]H)*5(-S-(-[:-15])(-[:45])-(<(=[::60]O)-[::-60]O^{-})--)-N-(=O)-))=-=-)} \ce{K^+}

这种物质叫做\href{https://www.chem960.com/cas/113984/}{青霉素钾},常用于制药。选取这个结构式来绘制是因为它具有许多有机化学结构式中会出现的特性,例如环、环嵌套以及各种分支。说真的,如果您只是用\LaTeX 来做化学作业,那我建议读者直接导入插图,使用chemfig绘制如此简单的一个结构式就花费了我不少时间来查阅资料。

\item algorithm2e生成的伪代码:

\begin{algorithm}[H]
	\SetKwFunction{algoBubbleSort}{BUBBLE-SORT}
	\SetKwFunction{algoSwap}{SWAP}
	\SetKwData{inputArr}{A}
	
	\KwIn{待排序数组$A[n]$,要求其类型强有序}
	
	\KwOut{排好序的数组,仍存放在$A[n]$中}
	
	\SetAlgoNoLine
	
	\Fn{\algoBubbleSort{$A[n]$}}{
		\For{$i \leftarrow 1$ \KwTo $n$}{
			\For{$j \leftarrow 1$ \KwTo $n-i$}{
				\If{$A[j] > A[j+1]$}{\algoSwap{$A[j]$, $A[j+1]$}}
			}
		}
	}
\end{algorithm}

这是一段冒泡排序的伪代码(pseudo-code),仅仅是对算法的描述,不能被实际编译运行。冒泡排序是一个相当经典和简单的排序算法,它每一趟循环把前/后若干个值中最大/小者放在最后/前(取决于你要怎样的排序结果,以及如何实现它),所以很显然,每一趟排序后必然已经有至少一个值被放在了它最终该在的位置上。\href{https://labs.xjtudlc.com/labs/wldmt/reading\%20list/books/Algorithms\%20and\%20optimization/Introduction\%20to\%20Algorithms.pdf}{《算法导论》}中具体讨论了这个算法。

\item listings提供的代码片功能:

\begin{lstlisting}[style = GNU_Cpp20_DevCpp]
// A simple C++ example
#include<iostream>
using std::cout;
using std::endl;

int main() {
cout << "Hello world" << endl;
return 0;
}
\end{lstlisting}

这是一个简单的C++程序,作用就是输出一行Hello world。一个常见的误解是“std::cin一定比scanf慢”,其实并不一定如此;SysTutorials上一个题为“\href{https://www.systutorials.com/is-cin-much-slower-than-scanf-in-c/}{C++的cin真的比scanf慢很多吗}”的主题讨论了这一点。

\item 函数图象:

\begin{figure}[H]
	\centering
	\begin{tikzpicture}
		\begin{axis}[
			xmin = -.5, xmax = 4,
			ymin = -1.2, ymax = 1.2,
			axis line style = ->,
			axis x line = middle,
			axis y line = middle,
			xlabel = {$x$},
			ylabel = {$sin\,2x$}]
			\addplot[no marks, blue, -]expression[domain=-.5:4, samples = 128]{sin(2*deg(x))};
			% adplot中的横杠参数表示绘出的曲线不带箭头
		\end{axis}
	\end{tikzpicture}
\end{figure}

这个例子展示了如何在\LaTeX 下用pgfplots绘制一幅笛卡尔坐标系下的函数图象。除此之外,pgf包还有许多其他绘图功能。

\end{enumerate}

\renewcommand{\thefootnote}{\arabic{footnote}}

\subsection{图表和图片宏包}

caption包和subcaption包用于题注。包含图片时,最常用的包是graphicx。multirow包用于进行单元格合并,\TikZ 和pgf用于绘制图样。制表时经常配合array包制定样式。特别地,circui\TikZ 包主要用于绘制电路图。

\begin{figure}[H]
	\centering
	\subcaptionbox{GNU logo\label{figure::GNUlogo}}
		{\includegraphics[height = 3cm, width = 3cm]{contents/assets/img/GNU_logo.png}}
	\subcaptionbox{Freedo\label{figure::Freedo}}
		{\includegraphics[height = 3cm, width = 3cm]{contents/assets/img/Freedo.png}}
	\caption{两个logos}\label{figure::two_logos}
\end{figure}

这个例子中,\LaTeX 为这两张图片分别分配了题注,GNU的logo得到编号\ref*{figure::GNUlogo},小企鹅Freedo\footnote{Freedo是\href{http://www.fsfla.org/ikiwiki/selibre/linux-libre/}{GNU Linux-libre}(一个操作系统内核)的吉祥物。值得一提的是,Linux最广为人知的吉祥物也是一只企鹅,名为Tux。然而,许多Linux发行版如今选择其他动物作为自身的吉祥物。}得到编号\ref*{figure::Freedo}。两张图还有共同题注,编号为\ref*{figure::two_logos}。Aur\'elio A. Heckert提供了图\ref*{figure::GNUlogo},Rub\'en Rodr\'iguez P\'erez等人提供了图\ref*{figure::Freedo}\footnote{这张图是在\url{https://svgtopng.com/}进行了SVG格式到PNG格式的转码后得到的。},这两张图都来自\href{https://www.gnu.org/graphics/graphics.html}{GNU画廊}\footnote{GNU是一个致力于发扬自由软件精神的组织,知名的GPL协议就来自于这个组织。作为最知名的开源支持者之一,GNU经常以相当直白的方式表达对专有软件的不满。包括CIA和亚马逊在内的组织和企业都被\href{https://www.gnu.org/gnu/gnu.html}{GNU操作系统官网}挂起来批评过。其他一些开源支持者,例如FFmpeg的开发者,还列出了一个“\href{https://github.com/FFmpeg/web/blob/master/src/shame}{耻辱柱}”,上面列出了未遵守开源协议的软件清单;许多大厂也没能逃过被钉上FFmpeg耻辱柱的命运。不仅如此,许多基本开源的软件(例如某些官方包管理服务器使用了专有软件的操作系统)因为被认为“不够开源”而被各类自由软件组织开除自由软件籍。}。

% C是自定义样式,需要array包支持并用\newcolumntype定义
\begin{table}[H]
	\centering
	\begin{tabular}{|C{2cm}|C{2cm}|C{2cm}|C{2cm}|} 
		\hline
		(0,0) & \multicolumn{2}{c|}{row 0} & ~ \\ \hline
		\multirow{2}*{column 0} & \multicolumn{2}{c|}{\multirow{2}*{center}} & ~ \\ \cline{4-4}
		~ & \multicolumn{2}{c|}{} & ~ \\ \hline
		~ & ~ & ~ & ~ \\ \hline
	\end{tabular}
	\caption{一张随意画的\LaTeX 表格}
\end{table}

这个表格例子展示了\LaTeX 中复杂表格的绘制方法。\LaTeX 下的原生表格实际上做不到这点,但是我们有许多支持自定义表格的宏包。一般而言,Microsoft Office Word能做到的表格功能,\LaTeX 都有对应实现。

\begin{figure}[H]
\centering
\begin{circuitikz}
	\draw (0,0) node[and4b](aA){\&};
	\draw (2,0) node[and4b](aB){\&};
	\draw
		(aA.in1) -- +(-.25,0)
		(aA.in2) -- +(-.25,0)
		(aA.in3) -- +(-.25,0)
		(aA.in4) -- +(-.25,0)
		(aB.in1) -- +(-.25,0)
		(aB.in2) -- +(-.25,0)
		(aB.in3) -- +(-.25,0)
		(aB.out) -- +(.25,0);
	\draw (aB.in4) node[ocirc, left]{}
		-- +(-.5,0) to +(-.5,0.75) -- (aA.out);
\end{circuitikz}
\end{figure}

这是一个使用circuit\TikZ 包绘制的电路图,包含两个连在一起的四路与门,其中上一级与门的输出反相后连接到下一级与门的第四个输入(注意后一级与门的i4引脚上的圆圈标记)。这个芯片形式的与门部件是使用pgf底层命令定义的,并在circuit\TikZ 包层面调用封装好的部件。可以在本文随附源码的contents/assets/circuitikz/and4b.tex看到这个部件的详细定义,包括绘图细节和针脚排布。

\begin{figure}[H]
	\centering
	\begin{tikzpicture}
		\begin{axis}[
			xmin = -1, xmax = 10,
			ymin = 0, ymax = 100,
			axis on top,
			xlabel = {题目编号},
			ylabel = {得分率(百分比)}]
			\addplot[ybar, bar width = 10pt, bar shift = 0pt,
				fill = blue, draw = black] plot coordinates{
				(0,34.7)(1,55.6)(2,23.1)(3,97.8)(4,33.4)
				(5,67.6)(6,95.2)(7,87.6)(8,5.7)(9,65.4)
				};
		\end{axis}
	\end{tikzpicture}
\end{figure}

% 制图表

% 排版布局类宏包





下文中将会逐步讲解\LaTeX 内置功能和宏包中的常用功能。

\caution{有许多宏包中的工具需要编译两次才能得到正确结果。所以,遇到编译没出错但效果样式有问题的情况先别着急,试试再编译一次。极少数情况下,这也有助于修正编译错误。发布前也要编译两次再进行审阅。此外,有时候宏包包含顺序也会引发问题,出现宏包内部错误时不妨尝试调整宏包的包含顺序。}